\documentclass[aspectratio=1609]{beamer}
\usetheme{Sii}

\author{Bogus�aw Klimas}
\title{Jenkins - Continuous Integration i nie tylko}
\date{01-02 wrzesie� 2018}

%\setbeameroption{show notes}
%\setbeamercolor{note page}{bg=white}

\begin{document}

% ===================================================== %
% Strona tytu�owa                                       %
% ===================================================== %
\begin{frame}
\titlepage
\end{frame}

% ===================================================== %
% Agenda                                                %
% ===================================================== %
\begin{frame}
\frametitle{Agenda}
\tableofcontents
\end{frame}

% ===================================================== %
% O nas                                                 %
% ===================================================== %
\section{O nas}
\begin{frame}\frametitle{O nas}
\begin{block}{O mnie}
 Programuje w Javie od 10 lat\\
 Jenkinsem zajmuje si� od $\sim8$ lat\\
 Sygnity, Sii (Sabre)\\
 Opiekun plugina git-parameter\footnote{https://github.com/jenkinsci/git-parameter-plugin}
\end{block}
\begin{block}{O Was}
    Ka�dy po kilka s��w
\end{block}
\end{frame}

% ===================================================== %
% Cel                                                 %
% ===================================================== %
\section{Cel szkolenia, warsztat�w}
\begin{frame}\frametitle{Cel szkolenia, warsztat�w}
\begin{block}{Cel !}
 Zdobycie wiedzy praktycznej z Jenkinsa\\
 Wdro�enie i zarz�dzanie Jenkinsem jak i "Jobami"
\end{block}
\end{frame}

% ===================================================== %
% Jenkins a co to                                       %
% ===================================================== %
\section{Jenkins a co to?}
\begin{frame}\frametitle{Jenkins a co to\footnote{https://jenkins.io/press/}}
\begin{block}{Elevator Pitch}
 Jenkins is an open source automation server which enables developers around the world to reliably build, test, and deploy their software.
\end{block}
\begin{block}{Blurb}
 Jenkins, originally founded in 2006 as "Hudson", is one of the leading automation servers available. 
 Using an extensible, plugin-based architecture developers have created hundreds of plugins to adapt Jenkins to a multitude of build, test, and deployment automation workloads. 
 In 2015, Jenkins surpassed 100,000 known installations making it the most widely deployed automation server.
\end{block}
\end{frame}

% ===================================================== %
% Jenkins kto za tym wszystkim stoi                     %
% ===================================================== %
\begin{frame}\frametitle{Jenkins kto za tym wszystkim stoi}
\begin{columns}
 \begin{column}{.30\textwidth}
  \begin{block}{Kohsuke Kawaguchi}
   \pgfimage[height=3.0cm]{kk}
  \end{block}
 \end{column}
 \begin{column}{.70\textwidth}
  \begin{block}{Links}
   \footnotesize
   \begin{itemize}
    \item https://jenkins.io
    \item https://groups.google.com/forum/\#!forum/jenkinsci-dev
    \item https://issues.jenkins-ci.org/
    \item https://wiki.jenkins-ci.org/display/JENKINS/Home
    \item https://stats.jenkins.io/
   \end{itemize}
  \end{block}
   \pgfimage[height=2.5cm]{CloudBees}
 \end{column}
\end{columns}
\end{frame}

% ===================================================== %
% Trudne pocz�tki                                       %
% ===================================================== %
\begin{frame}\frametitle{Pocz�tki bywaj� trudne}
   \pgfimage[height=6.0cm]{begin}
\end{frame}


% ===================================================== %
% Uruchomienie, instalacja                              %
% ===================================================== %
\section{Uruchomienie, instalacja}
\begin{frame}[fragile]\frametitle{Uruchomienie, instalacja}
\begin{itemize}
 \item Instalacja z repozytorium
  \begin{verbatim}
  # dnf -y install jenkins
  # systemctl enable jenkins.service
  # systemctl start jenkins.service
  JENKINS_HOME=/var/lib/jenkins
  binaria: /usr/share/jenkins
  log: /var/log/jenkins/jenkins.log
  \end{verbatim}
 \item Uruchomienie z war'a
  \begin{verbatim}
  $ java -jar jenkins.war
  JENKINS_HOME=$HOME/.jenkins
  
  $./jenkins.sh start|stop
  \end{verbatim}
\end{itemize}
\end{frame}

% ===================================================== %
% Administracja                                         %
% ===================================================== %
\section{Administracja}
\begin{frame}\frametitle{Administracja vel Manage Jenkins}
\begin{itemize}
 \item About Jenkins
 \item System Information
 \item Manage Old Data
 \item Reload Configuration from Disk
 \item Load statistics
 \item System Log
 \item Manage Plugins
 \item Security
 \item Configure system
 \item Global Tool Configuration
 \item ...
\end{itemize}
\end{frame}

% ===================================================== %
% Trudne pocz�tki                                       %
% ===================================================== %
\begin{frame}\frametitle{LDAP}
   \pgfimage[height=6.0cm]{ldap_jenkins}
\end{frame}

% ===================================================== %
% Slave                                                 %
% ===================================================== %
\section{Jenkins slaves}
\begin{frame}\frametitle{Jenkins slaves}
 
\end{frame}

% ===================================================== %
% Jenkins & groovy                              %
% ===================================================== %
\section{Jenkins \& groovy}
\begin{frame}\frametitle{Jenkins \& groovy \sout{czyli Scriptler}}
 \begin{block}{Javadoc}
  https://javadoc.jenkins.io/
 \end{block}
\end{frame}

% ===================================================== %
% Jobs                                                 %
% ===================================================== %
\section{Jenkins jobs}
\begin{frame}\frametitle{Jenkins jobs}
 \begin{itemize}
  \item Freestyle project
  \item Maven project
  \item Pipeline
 \end{itemize}
\end{frame}

% ===================================================== %
% Views                                                 %
% ===================================================== %
% \section{Jenkins views}
% \begin{frame}\frametitle{Jenkins views}
%  \begin{itemize}
%   \item Build Pipeline View
%   \item List View
%  \end{itemize}
% \end{frame}

% ===================================================== %
% Dashboard                                                 %
% ===================================================== %
\section{Jenkins dashboard}
\begin{frame}\frametitle{Jenkins dashboard}
 \begin{itemize}
  \item Build Monitor View
 \end{itemize}
\end{frame}

% ===================================================== %
% Calculator plugin                                     %
% ===================================================== %
\section{Plugins}
\begin{frame}\frametitle{Plugins}
 \begin{itemize}
  \item Hallo World
  \item Need Beer
  \item Calculator
%  \item DirDigger
 \end{itemize}
\end{frame}


% ===================================================== %
% Pytania                                               %
% ===================================================== %
\section{Pytania}
\begin{frame}{Pytania}
\vspace*{10mm}
\begin{center}
 \LARGE{Pytania?}
\end{center}
\vspace*{30mm}
\begin{flushright}
 \LARGE Dzi�kuje! 
\end{flushright}
\end{frame}
\end{document}
